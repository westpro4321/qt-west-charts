\documentclass[11pt,twoside,a4paper,final]{article}
\usepackage[utf8]{inputenc}
\usepackage[T1]{fontenc}
\usepackage[MeX]{polski}
\usepackage{graphicx}
\usepackage{url}
\usepackage{hyperref}
\bibliographystyle{splncs}

\begin{document}

\date{14 maja 2013}
\title{Opis wymagań}

\author{Łukasz Szewczyk}

\maketitle

\section{Wymagania}
,,Określenie wymagań, jakie musi spełniać oprogramowanie, jest tym miejscem projektu, w którym najbardziej i~najwyraźniej stykają się interesy wszystkich jego udziałowców.''~\cite{sacha-wymagania}

\section{Wymagania funkcjonalne}
\subsection{Wykres liniowy}
\subsubsection{Tworzenie wykresu}
Biblioteka musi umożliwiać utworzenie wykresu liniowego prezentującego dane z~jednej bądź kilku serii danych. Na wykres liniowy składają się:
\begin{itemize}
\item{układ współrzędnych, zbudowany na dwóch osiach -- jednej poziomej, odpowiadającej dziedzinie i jednej pionowej, odpowiadającej zbiorowi wartości,}
\item{serie danych,}
\item{krzywe łamane prezentujące dane z~serii dostarczonych do wykresu, posiadające nazwę, domyślnie pobieraną z serii danych doń przypisanej, oraz unikalny w~danym wykresie kolor,}
\item{siatka, czyli zbiór prostopadłych linii dopasowanych do osi, ułatwiających szybkie, wzrokowe porównywanie danych,}
\item{tytuł wykresu,}
\item{tytuły osi,}
\item{legenda,}
\item{tło.}
\end{itemize}

Wyświetlanie każdego z~elementów powinno być opcjonalne. 
\subsubsection{Modyfikacja stanu wykresu}
Już po utworzeniu wykresu powinny być możliwe do skonfigurowania następujące parametry:
\begin{itemize}
\item{zawartość serii danych,}
\item{kolor krzywej,}
\item{nazwa krzywej (serii danych),}
\item{opis osi,}
\item{kolor siatki,}
\item{ziarnistość siatki,}
\item{kolor/grafika tła.}
\end{itemize}

\subsubsection{Dostępne operacje}
Operacje, które powinny być dostępne dla wykresu liniowego:
\begin{itemize}
\item{przybliżanie i oddalanie,}
\item{zaznaczanie przedziałów dziedziny,}
\item{zaznaczanie pojedynczej próbki danych,}
\item{zaznaczenie krzywej.}
\end{itemize}

\subsection{Wykres słupkowy}
\subsubsection{Tworzenie wykresu}
Biblioteka musi umożliwiać utworzenie wykresu słupkowego prezentującego dane z~jednej bądź kilku serii danych. Wykres ten ma przyjmować jedną z~dwóch orientacji: horyzontalną bądź wertykalną. Wertykalny wykres słupkowy jest czasem nazywany wykresem kolumnowym. Na wykres słupkowy składają się:
\begin{itemize}
\item{układ współrzędnych, zbudowany na dwóch osiach -- jednej poziomej i jednej pionowej, z których jedna odpowiada dyskretnej dziedzinie, a inna ciągłemu zbiorowi wartości,}
\item{serie danych,}
\item{prostokątne słupki o długości proporcjonalnej do reprezentowanej wartości. Kolor wszystkich słupków z~danej serii danych jest jednakowy i unikalny w skali wykresu,}
\item{siatka, czyli zbiór prostopadłych linii dopasowanych do osi, ułatwiających szybkie porównywanie danych,}
\item{tytuł wykresu,}
\item{tytuły osi,}
\item{legenda,}
\item{tło.}
\end{itemize}

Wyświetlanie każdego z~elementów powinno być opcjonalne. 
\subsubsection{Modyfikacja stanu wykresu}
Już po utworzeniu wykresu powinny być możliwe do skonfigurowania następujące parametry:
\begin{itemize}
\item{zawartość serii danych,}
\item{kolor grupy słupków (serii danych),}
\item{nazwa grupy słupków (serii danych),}
\item{opis osi,}
\item{kolor siatki,}
\item{ziarnistość siatki,}
\item{kolor/grafika tła.}
\end{itemize}

\subsubsection{Dostępne operacje}
Operacje, które powinny być dostępne dla wykresu słupkowego:
\begin{itemize}
\item{przybliżanie i oddalanie,}
\item{zaznaczanie przedziałów dziedziny,}
\item{zaznaczanie pojedynczego słupka,}
\item{zaznaczenie grupy słupków.}
\end{itemize}


\subsection{Wykres kołowy}
\subsubsection{Tworzenie wykresu}
Biblioteka musi umożliwiać utworzenie wykresu kołowego prezentującego dane z~jednej serii danych. Na wykres kołowy składają się:
\begin{itemize}
\item{serie danych}
\item{wycinki kołowe o~wspólnym środku, których kąt środkowy jest proporcjonalny do reprezentowanej wartości. Suma kątów środkowych wszystkich wycinków wynosi 360 stopni. Kolor danego wycinka jest unikalny w skali wykresu.}
\item{tytuł wykresu,}
\item{opis próbek,}
\item{legenda,}
\item{tło.}
\end{itemize}

Wyświetlanie każdego z~elementów powinno być opcjonalne. 

\subsubsection{Modyfikacja stanu wykresu}
Już po utworzeniu wykresu powinny być możliwe do skonfigurowania następujące parametry:
\begin{itemize}
\item{zawartość serii danych,}
\item{kolor wycinka (próbki danych),}
\item{tytuł próbki,}
\item{opis próbki,}
\item{kolor/grafika tła.}
\end{itemize}

\subsubsection{Dostępne operacje}
Operacje, które powinny być dostępne dla wykresu kołowego:
\begin{itemize}
\item{obracanie wykresu,}
\item{wysuwanie wycinka.}
\end{itemize}

\subsection{Elementy składowe wykresu}
\subsubsection{Serie danych}
Biblioteka musi zapewniać uniwersalny interfejs do danych oraz obsługiwać serie danych. Operacje dodawania, modyfikowania i~usuwania próbek danych powinny być możliwie proste dla użytkownika
\subsubsection{Legenda}
Dla wykresów obsługujących wiele serii danych, powinna prezentować kolory oraz tytuły tych serii.
Natomiast dla wykresów jednoseryjnych prezentowana powinna być informacja o kolorze i tytule każdej z próbek.
\subsubsection{Siatka}
Element składający się ze zbioru prostopadłych linii, ułatwiający porównywanie danych różnych próbek. Powinna istnieć możliwość określenia grubości i~koloru linii oraz ziarnistości samej siatki.
\subsubsection{Osie}	
\subsubsection{Tło}
Tłem wykresu może być kolor, gradient bądź obrazek.

\subsection{Reaktywność}
TODO: DUPA!!!
Wykresy muszą umożliwiać reakcję na wszelkie zdarzenia, np. zawieszenie kursora na wycinku wykresu kołowego czy wciśnięcie klawisza na klawiaturze.

\subsection{Interfejsy}
Biblioteka musi umożliwiać tworzenie wykresów zarówno w~klasycznym Qt jak i~Qt~Quick~2. 

\subsubsection{Interfejs C++}
Interfejs dla klasycznego Qt powinien możliwie dobrze wpisywać się w styl Qt opisany w artykule~\cite{qt-style-API}.

\subsubsection{Interfejs QML}
Interfejs ten powinien spełniać paradygmaty programowania deklaratywnego.



\subsection{Efekty graficzne}
Wszystkie opisane tu funkcjonalności są opcjonalne, a~ich realizacja nie jest konieczna do zakończenia prac nad biblioteką.

\subsubsection{Przestrzenność}
Biblioteka może w~jakiś sposób nadawać wykresom głębi i~wrażenia przestrzenności (2,5D). 

\subsubsection{Motywy}
Dodatkiem, który podniósłby atrakcyjność wykresów jest wysokopoziomowy mechanizm motywów, podobny do \textit{QStyle}~\cite{qstyle}. Z~jego pomocą, tworzenie zbioru wykresów o jednolitej kolorystyce i czcionkach stałoby się łatwe. Zmiana motywu dla danego wykresu powinna sprowadzać się do prostej operacji.

\subsubsection{Animacje}
Animacje procesu tworzenia lub modyfikowania wykresu -- jest to element nadający aplikacjom dynamizmu.

\subsubsection{Generowanie obrazków}
Łatwym powinno być generowanie plików graficznych na podstawie istniejących wykresów, np. w~formatach .PNG i~.SVG.



\section{Wymagania pozafunkcjonalne}


\subsection{Struktura biblioteki}
Biblioteka powinna mieć przejrzystą strukturę, która w~połączeniu z~dokumentacją w~stylu Qt (przykład tutaj~\cite{qt-doc}) powinna umożliwić programistom sprawne przeanalizowanie jej działania i~szybkie przystąpienie do tworzenia wykresów.

\subsection{Wymienność biblioteki}
Biblioteka powinna wykorzystywać mechanizmy pozwalające na tworzenie wymiennych bibliotek dynamicznych. Wprowadzenie nowej wersji biblioteki z~niezmienionym interfejsem nie powinno wymagać przebudowania całej aplikacji.

\subsection{Nowoczesność i~uniwersalność}
Biblioteka powinna wykorzystywać możliwie nowe technologie, m.in. Qt5. Jednak użycie standardu \textit{C++11} nie jest wskazane ze względu na ograniczenie liczby potencjalnych odbiorców. Komponenty dostarczane do użytku programistom powinny być możliwie wysokopoziomowe i~uniwersalne w~użyciu.

\subsection{Wydajność}
Jako, że okoliczności wykorzystania wykresów biurowych są znacząco inne niż wykresów technicznych oraz natura ich danych jest dużo bardziej statyczna, optymalizacja nie jest tu kwestią najważniejszą. 

\subsubsection{Wydajność wykresów w uk. współrzędnych}
Wstępnie zakłada się, że operacje dokonywane na wykresach osadzonych w~układzie współrzędnych powinny działać względnie płynnie dla danych do tysiąca próbek. Nie jest wykluczona późniejesza optymalizacja
pozwalająca komfortowo operować na większych ilościach danych.

\subsubsection{Wydajność wykresu kołowego}
Wykres kołowy, nie powinien powodować ,,zamrażania aplikacji'' przy liczbie próbek nie~przekraczającej 100. 

\subsection{Niezawodność}
Jak już zostało wcześniej stwierdzone, natura oraz zastosowania wykresów biurowych różnią się od technicznych, a~co za tym idzie, mają również inne wymagania dotyczące niezawodności. Przewiduje się, że biblioteka będzie przeznaczona do aplikacji finansowych i~biurowych, a nie systemów czasu rzeczywistego. Jednakowoż w celu minimalizacji liczby błędów w kodzie, powinny zostać zastosowane testy, m.in. jednostkowe, regresyjne, itp. Zachowanie biblioteki w warunkach ekstremalnych, np. przepełnienia stosu, nie jest głównym celem projektu.

%\subsubsection{Wiarygodnosć}

\subsection{Skalowalność}
Zarówno dodawanie nowych jak i~usuwanie już istniejących elementów biblioteki powinno być łatwe i~nie powinno mieć wpływu na stabilność pracy biblioteki. Dodawanie nowych elementów powinno być możliwe dzięki uniwersalnym interfejsom. Natomiast usuwanie istniejących elementów powinno sprowadzać się do wyłączenia ich z~procesu kompilacji biblioteki.

\begin{thebibliography}{}
\bibitem[1]{sacha-wymagania}
Inżynieria oprogramowania -- rozdział 2, Krzysztof Sacha, Wydawnictwo Naukowe PWN, 2010, ISBN: 978-83-01-16179-8
\bibitem[2]{qt-style-API}
API w stylu Qt \url{http://doc.qt.digia.com/qq/qq13-apis.html}
\bibitem[3]{qstyle}
QStyle \url{http://doc.qt.digia.com/4.7-snapshot/qstyle.html#details}
\bibitem[4]{qt-doc}
Przykład dokumentacji Qt \url{http://qt-project.org/doc/qt-4.8/qobject.html}

\end{thebibliography}

\end{document}
