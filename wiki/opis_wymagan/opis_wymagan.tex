\documentclass[11pt,twoside,a4paper,final]{article}
\usepackage[utf8]{inputenc}
\usepackage[T1]{fontenc}
\usepackage[MeX]{polski}
\usepackage{graphicx}
\usepackage{url}
\usepackage{hyperref}
\bibliographystyle{splncs}

\begin{document}

\date{13 lutego 2013}
\title{Opis wymagań}

\author{Łukasz Szewczyk}

\maketitle

\section{Wymagania}
,,Określenie wymagań, jakie musi spełniać oprogramowanie, jest tym miejscem projektu, w którym najbardziej i~najwyraźniej stykają się interesy wszystkich jego udziałowców.''~\cite{sacha-wymagania}

\section{Wymagania funkcjonalne}

\subsection{Tworzenie wykresów}

\subsubsection{Podstawowe wykresy 2D}
Konieczne jest udostępnienie najpopularniejszych wykresów biurowych, czyli: słupkowego, liniowego i~kołowego. 

\subsubsection{Mniej standardowe wykresy}
Dodatkowo powinno być możliwe stopniowe uzupełnianie biblioteki o~kolejne typy wykresów, takie jak: bąbelkowy, punktowy czy pierścieniowy.

\subsubsection{Łączenie wykresów}
Powinno być możliwe tworzenie w jednym, wspólnym układzie współrzędnych kilku różnych wykresów, np. słupkowego i~liniowego.

\subsubsection{Elementy dekoracyjne}
Biblioteka musi udostępniać takie elementy wykresów jak legenda, siatka, tło i~osie, przy czym ich wyświetlaniem powinien sterować programista korzystający z biblioteki. Wszelkie właściwości wyświetlanych elementów reprezentujących próbki danych, takie jak kolor czy podpis, muszą być intuicyjne w obsłudze i~przyjmować sensowne domyślne wartości.

\subsection{Interakcja}
Wykresy tworzone za pomocą biblioteki muszą być nastawione na interakcję z użytkownikiem.

\subsubsection{Reaktywność}
Wykresy muszą umożliwiać reakcję na wszelkie zdarzenia, np. zawieszenie kursora na wycinku wykresu kołowego czy wciśnięcie klawisza na klawiaturze.

\subsubsection{Wykresy w uk. współrzędnych} 
Dla wykresów osadzonych w kartezjańskim układzie współrzędnych powinno być możliwe zaznaczanie i przybliżanie ich fragmentów. 

\subsubsection{Wykresy inne} 
Dla pozostałych wykresów powinne być dostępne operacje zgodne z ich naturą, np. dla wykresów kołowego i pierścieniowego powinna istnieć opcja obracania ich. 


\subsection{Interfejsy}
Biblioteka musi umożliwiać tworzenie wykresów zarówno w~klasycznym Qt jak i~Qt~Quick~2. 

\subsubsection{Interfejs C++}
Interfejs dla klasycznego Qt powinien możliwie dobrze wpisywać się w styl Qt opisany w artykule~\cite{qt-style-API}.

\subsubsection{Interfejs QML}
Interfejs ten powinien spełniać paradygmat programowania deklaratywnego.

\subsection{Dane}

\subsubsection{Uniwersalny interfejs}
Biblioteka musi zapewniać uniwersalny interfejs do danych oraz obsługiwać serie danych. Operacje dodawania, modyfikowania i~usuwania próbek danych powinny być możliwie proste dla użytkownika.

\subsubsection{Implementacja interfejsu} Opcjonalna jest implementacja dla danych pobieranych w formacie XML lub JSON.

\subsection{Osie}

\subsubsection{Liczba osi}
Podstawową funkcjonalnością jest obsługa dwóch osi -- jednej pionowej i~jednej poziomej. Jednak implementacja nie powinna wykluczać obługi większej liczby osi. 

\subsubsection{Nieliniowość wykresu}
Uniwersalność interfejsów powinna umożliwiać wprowadzenie w~późniejszym czasie osi innych niż liniowa.

\subsection{Efekty graficzne}
Wszystkie opisane tu funkcjonalności są opcjonalne, a~ich realizacja nie jest konieczna do zakończenia prac nad biblioteką.

\subsubsection{Przestrzenność}
Biblioteka może w~jakiś sposób nadawać wykresom głębi i~wrażenia przestrzenności. 

\subsubsection{Motywy}
Dodatkiem, który podniósłby atrakcyjność wykresów jest wysokopoziomowy mechanizm motywów, podobny do \textit{QStyle}~\cite{qstyle}. Z~jego pomocą, tworzenie zbioru wykresów o jednolitej kolorystyce i czcionkach stałoby się łatwe. Zmiana motywu dla danego wykresu powinna sprowadzać się do prostej operacji.

\subsubsection{Animacje}
Ostatnim elementem może być możliwość animacji procesu tworzenia wykresu -- jest to element nadający aplikacjom dynamizmu. 

\subsubsection{Generowanie obrazków}
Łatwym powinno być generowanie plików graficznych na podstawie istniejących wykresów, np. w~formatach .PNG i~.SVG.

\subsection{Przenośność}
Biblioteka musi wpisywać się w politykę Qt brzmiącą: \textit{pisz raz, kompiluj wielokrotnie}. Musi być przenośna na najpopularniejszych, wspieranych przez Qt platformach. Minimum to uruchomienie na platformach:

\subsubsection{Windows NT}
\subsubsection{Linux (Ubuntu)}


\section{Wymagania pozafunkcjonalne}


\subsection{Struktura biblioteki}
Biblioteka powinna mieć przejrzystą strukturę, która w~połączeniu z~dokumentacją w~stylu Qt (przykład tutaj~\cite{qt-doc}) powinna umożliwić programistom sprawne przeanalizowanie jej działania i~szybkie przystąpienie do tworzenia wykresów.

\subsection{Wymienność biblioteki}
Biblioteka powinna wykorzystywać mechanizmy pozwalające na tworzenie wymiennych bibliotek dynamicznych. Wprowadzenie nowej wersji biblioteki z~niezmienionym interfejsem nie powinno wymagać przebudowania całej aplikacji.

\subsection{Nowoczesność i~uniwersalność}
Biblioteka powinna wykorzystywać możliwie nowe technologie, m.in. Qt5. Jednak użycie standardu \textit{C++11} nie jest wskazane ze względu na ograniczenie liczby potencjalnych odbiorców. Komponenty dostarczane do użytku programistom powinny być możliwie wysokopoziomowe i~uniwersalne w~użyciu.

\subsection{Wydajność}
Jako, że okoliczności wykorzystania wykresów biurowych są znacząco inne niż wykresów technicznych oraz natura ich danych jest dużo bardziej statyczna, optymalizacja nie jest tu kwestią najważniejszą. 

\subsubsection{Wykresy w uk. współrzędnych}
Wstępnie zakłada się, że operacje dokonywane na wykresach osadzonych w~układzie współrzędnych powinny działać względnie płynnie dla danych do tysiąca próbek. Nie jest wykluczona późnijesza optymalizacja
pozwalająca operować na większych ilościach danych.

\subsubsection{Wykresy inne}
Wykresy inne, np. kołowy, nie powinny powodować ,,zamrażania aplikacji'' przy liczbie rekordów nie~przekraczającej 100. 


\begin{thebibliography}{}
\bibitem[1]{sacha-wymagania}
Inżynieria oprogramowania -- rozdział 2, Krzysztof Sacha, Wydawnictwo Naukowe PWN, 2010, ISBN: 978-83-01-16179-8
\bibitem[2]{qt-style-API}
API w stylu Qt \url{http://doc.qt.digia.com/qq/qq13-apis.html}
\bibitem[3]{qstyle}
QStyle \url{http://doc.qt.digia.com/4.7-snapshot/qstyle.html#details}
\bibitem[4]{qt-doc}
Przykład dokumentacji Qt \url{http://qt-project.org/doc/qt-4.8/qobject.html}

\end{thebibliography}

\end{document}
