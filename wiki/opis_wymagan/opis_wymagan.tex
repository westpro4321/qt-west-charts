\documentclass[11pt,twoside,a4paper,final]{llncs}
\usepackage[utf8]{inputenc}
\usepackage[T1]{fontenc}
\usepackage[MeX]{polski}
\usepackage{graphicx}
\usepackage{url}
\usepackage{hyperref}
\bibliographystyle{splncs}

\begin{document}

\date{13 lutego 2013}
\title{Opis wymagań}

\author{Łukasz Szewczyk}
\institute{Politechnika Warszawska, Wydział Elektroniki i Technik Informacyjnych}
\maketitle

\section{Wymagania}
,,Określenie wymagań, jakie musi spełniać oprogramowanie, jest tym miejscem projektu, w którym najbardziej i najwyraźniej stykają się interesy wszystkich jego udziałowców."~\cite{sacha-wymagania}

\section{Wymagania funkcjonalne}

\subsection{Tworzenie wykresów}
Biblioteka musi umożliwiać łatwe tworzenie wykresów 2D. Konieczne jest udostępnienie najpopularniejszych wykresów biurowych, czyli: słupkowego, liniowego i kołowego. Dodatkowo powinno być możliwe stopniowe uzupełnianie biblioteki o kolejne typy wykresów, takie jak: bąbelkowy, punktowy czy pierścieniowy. Powinno być możliwe tworzenie kilku wykresów w jednym układzie współrzędnych, np. słupkowy i liniowy.
Biblioteka musi udostępniać takie elementy wykresów jak legenda, siatka i osie, przy czym ich wyświetlaniem powinien sterować programista korzystający z biblioteki.

\subsection{Interakcja}
Wykresy tworzone za pomocą biblioteki muszą być nastawione na interakcję z użytkownikiem, i tak dla wykresów osadzonych w kartezjańskim układzie współrzędnych, powinno być możliwe zaznaczanie i przybliżanie ich fragmentów. Z kolei dla wykresów kołowego i pierścieniowego powinna istnieć opcja obracania ich. Wykresy muszą umożliwiać reakcję na ruchy kursora myszy po ich powierzchni, np. zawieszenie kursora na wycinku wykresu kołowego.

\subsection{Dane}
Biblioteka musi zapewniać uniwersalny interfejs do danych oraz obsługiwać serie danych. Opcjonalna jest implementacja dla danych pobieranych w formacie XML lub JSON. Biblioteka musi umożliwiać łatwe dodawanie, modyfikowanie i~usuwanie próbek danych.

\subsection{Osie}
Podstawową funkcjonalnością jest obsługa dwóch osi, jednej pionowej i jednej poziomej. Jednak implementacja nie powinna wykluczać obługi większej liczby osi. Ponadto ogólność interfejsów powinna umożliwiać wprowadzenie w~późniejszym czasie osi innych niż liniowa.

\subsection{Efekty graficzne}
Wszystkie opisane tu funkcjonalności są opcjonalne, a ich realizacja nie jest konieczna do zakończenia prac nad biblioteką.\newline
Biblioteka może udostępniać \textit{efekt 2,5D}. Kolejnym ciekawym dodatkiem, który podniósłby atrakcyjność wykresów jest wysokopoziomowy mechanizm motywów, podobny do \textit{QStyle}~\cite{qstyle}. Ostatnim dodatkiem może być możliwość animacji procesu tworzenia wykresu -- jest to element nadający aplikacjom dynamizmu. Łatwym powinno być generowanie plików graficznych na podstawie istniejących wykresów, np. w formatach .PNG i~.SVG.


\section{Wymagania pozafunkcjonalne}
\subsection{Obligatoryjne}
Biblioteka musi:
\begin{itemize}
\item{być przenośna -- minimum to działanie na platformach Windows i Ubuntu,}
\item{pozwalać na tworzenie atrakcyjnie i nowocześnie wyglądających wykresów 2D,}
\item{posiadać interfejs C++ w stylu Qt~\cite{qt-style-API},}
\item{posiadać interfejs dla Qt Quick.}
\end{itemize}

\subsection{Zalecane}
Biblioteka powinna:
\begin{itemize}
\item{wymagać do działania jedynie biblioteki standardowej C++ oraz bibliotek Qt,}
\item{wykorzystywać Qt w wersji 5,}

\end{itemize}


\begin{thebibliography}{}
\bibitem[1]{sacha-wymagania}
Inżynieria oprogramowania -- rozdział 2, Krzysztof Sacha, Wydawnictwo Naukowe PWN, 2010, ISBN: 978-83-01-16179-8
\bibitem[2]{qstyle}
QStyle \url{http://doc.qt.digia.com/4.7-snapshot/qstyle.html#details}
\bibitem[3]{qt-style-API}
API w stylu Qt \url{http://doc.qt.digia.com/qq/qq13-apis.html}

\end{thebibliography}

\end{document}
