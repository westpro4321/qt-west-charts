\documentclass[11pt,twoside,a4paper,final]{llncs}
\usepackage[utf8]{inputenc}
\usepackage[T1]{fontenc}
\usepackage[MeX]{polski}
\usepackage{graphicx}
\usepackage{url}
\usepackage{hyperref}
\bibliographystyle{splncs}

\begin{document}

\date{12 lutego 2013}
\title{Opis wymagań}

\author{Łukasz Szewczyk}
\institute{Politechnika Warszawska, Wydział Elektroniki i Technik Informacyjnych}
\maketitle

\section{Wymagania}
,,Określenie wymagań, jakie musi spełniać oprogramowanie, jest tym miejscem projektu, w którym najbardziej i najwyraźniej stykają się interesy wszystkich jego udziałowców."~\cite{sacha-wymagania}

\section{Wymagania funkcjonalne}
\subsection{Obligatoryjne}
Biblioteka musi umożliwiać:
\begin{itemize}
\item{tworzenie wykresów -- wymagane jest udostępnienie najpopularniejszych wykresów biurowych, czyli: słupkowego, liniowego i kołowego,}
\item{modyfikację danych -- dodawanie, edycję i usuwanie danych prezentowanych za pomocą wykresu,}
\item{wyświetlenie elementów dekoracyjnych wykresu, takich jak legenda, siatka czy osie,}
\item{interakcję docelowego użytkownika z wykresem -- zaznaczanie i przybliżanie fragmentu wykresu,}
\item{wyświetlanie wykresów różnych typów w jednym układzie współrzędnych.}
\end{itemize}

\subsection{Opcjonalne}
Biblioteka może udostępniać:
\begin{itemize}
\item{tworzenie mniej popularnych wykresów, takich jak: bąblekowy, punktowy czy pierścieniowy,}
\item{pobieranie danych do wyświetlania z plików XML,}
\item{rozszerzenie interakcji z użytkownikiem, np. poprzez wyświetlanie dodatkowych informacji o wskazywanej kursorem próbce danych,}
\item{generowanie plików SVG na podstawie istniejącego wykresu,}
\item{animację procesu budowania wykresu,}
\item{tworzenie nowych i wykorzystywanie istniejących motywów -- potrzeba stworzenia mechanizmu podobnego do \textit{QStyle}~\cite{qstyle}.}
\end{itemize}

\section{Wymagania pozafunkcjonalne}
\subsection{Obligatoryjne}
Biblioteka musi:
\begin{itemize}
\item{być przenośna -- minimum to działanie na platformach Windows i Ubuntu,}
\item{pozwalać na tworzenie atrakcyjnie i nowocześnie wyglądających wykresów 2D,}
\item{posiadać interfejs C++ w stylu Qt~\cite{qt-style-API},}
\item{posiadać interfejs dla Qt Quick.}
\end{itemize}

\subsection{Zalecane}
Biblioteka powinna:
\begin{itemize}
\item{wymagać do działania jedynie biblioteki standardowej C++ oraz bibliotek Qt,}
\item{wykorzystywać Qt w wersji 5,}

\end{itemize}


\begin{thebibliography}{}
\bibitem[1]{sacha-wymagania}
Inżynieria oprogramowania, Krzysztof Sacha, Wydawnictwo Naukowe PWN, 2010, ISBN: 978-83-01-16179-8
\bibitem[2]{qstyle}
QStyle \url{http://doc.qt.digia.com/4.7-snapshot/qstyle.html#details}
\bibitem[3]{qt-style-API}
API w stylu Qt \url{http://doc.qt.digia.com/qq/qq13-apis.html}

\end{thebibliography}

\end{document}
