\documentclass[11pt,twoside,a4paper,final]{article}
\usepackage[utf8]{inputenc}
\usepackage[T1]{fontenc}
\usepackage[MeX]{polski}
\usepackage{graphicx}
\usepackage{url}
\usepackage{hyperref}
\bibliographystyle{splncs}

\begin{document}

\date{21 maja 2013}
\title{Analiza wymagań}

\author{Łukasz Szewczyk}

\maketitle

\section{Wstęp}
Analiza wymagań zostanie zapoczątkowana diagramem przypadków użycia, zawierającym wszystkie najważniejsze wymagania wysokiego poziomu moją bibliotekę. Następnie wszystkie wymagające tego wymagania zostaną rozpisane w~sposób szczegółowy oraz, jeśli zajdzie taka potrzeba, podparte odpowiednim diagramem przypadków użycia.

\section{Diagram ,,poziomu zero''}

\section{Wymagania}


\begin{thebibliography}{}
\bibitem[1]{sacha-wymagania}
Inżynieria oprogramowania -- rozdział 2, Krzysztof Sacha, Wydawnictwo Naukowe PWN, 2010, ISBN: 978-83-01-16179-8
\bibitem[2]{qt-style-API}
API w stylu Qt \url{http://qt-project.org/wiki/API-Design-Principles}
\end{thebibliography}

\end{document}
