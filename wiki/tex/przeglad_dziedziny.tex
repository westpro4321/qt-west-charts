

\chapter{Przegląd dziedziny}
\section{Wstęp}
Przegląd ten zawiera opis kilku wybranych bibliotek służących do tworzenia wykresów. Zostały tu opisane biblioteki przeznaczone dla Qt i~C++, podzielone na rozwiązania komercyjne oraz darmowe. Dla każdej biblioteki zostały podane jej zalety i~wady. Następnie przedstawiam wykaz typów wykresów dostępnych we wszystkich opisanych produktach. Kolejne dwa paragrafy to zestawienie elementów wspólnych i~unikalnych. Ostatni fragment to podsumowanie zawierające wnioski płynące z~tego przeglądu.

\section{Rozwiązania komercyjne}
\subsection{Qt Commercial Charts}
Biblioteka ta została stworzona przez aktualnego właściciela Qt, czyli firmę Digia~\cite{digia}. Oferuje ona programiście szeroki wybór wykresów. Nie wymaga zagłębiania się w~swoją wewnętrzną budowę, a~tworzenie prostych wykresów polega na połączeniu kilku wysokopoziomowych obiektów takich jak seria czy widget widoku.\newline

Cała biblioteka opiera się na frameworku GraphicsView~\cite{graphicsview}, który wprowadza podział na scenę oraz widok, gdzie scena jest kontenerem na elementy prezentowane za pomocą widoku.
Biblioteka udostępnia dwa uniwersalne widoki. Zależnie od zastosowania można użyć obiektu klasy dziedziczącej z~QGraphicsView bądź z~QGraphicsWidget. Implementacje konkretnych wykresów zarządzają elementami dodawanymi do sceny.\newline

Biblioteka ta, jako jedyna z tutaj opisanych, udostępnia wszystkie swoje wykresy w~języku QML, zarówno dla QtQuick~1~i~2.
Jak sam producent zaznacza, silne uzależnienie biblioteki od GraphicsView sprawia, że osiąga ona lepsze wyniki wydajnościowe dla QtQuick~1.\newline

\textbf{Zalety:}
\begin{itemize}
\item{szeroki wybór wykresów,}
\item{operowanie na komponentach wysokiego poziomu,}
\item{interaktywność wszystkich elementów wykresu,}
\item{system motywów umożliwiający tworzenie wykresów o~spójnej kolorystyce,}
\item{obiekty mapujące dane ze standardowych modeli Qt do serii danych,}
\item{wtyczka dla Designera,}
\item{silne wsparcie dla QML.}\newline
\end{itemize}

\textbf{Wady:}
\begin{itemize}
\item{spadek uniwersalności poprzez uzależnienie prezentacji wykresów od konkretnych widoków,}
\item{wykorzystanie GraphicsView zamiast SceneGraph, powodujące niższą wydajność w QtQuick~2.}
\end{itemize}


\subsection{KD Charts}
Jest to biblioteka stworzona przez firmę KDAB~\cite{kdab}, specjalizującą się w~tworzeniu oprogramowanie w~Qt oraz prowadzeniu szkoleń dla programistów Qt. KD Charts w~odróżnieniu od poprzedniej biblioteki nie korzysta z~GraphicsView, a~z~mechanizmów niższego poziomu. Rysowanie odbywa się tu za pomocą systemu Arthur~\cite{arthur}, a do pisania wykorzystano silnik Scribe~\cite{scribe}.
Dzięki wykorzystaniu technologii niższego poziomu, programistom KDAB udało się uzyskać lepszą wydajność, szczególnie ważną przy oferowanych przez nich wykresach czasu rzeczywistego.\newline

KD Charts separuje dane od warstwy prezentacji wykorzystując modele znane z~popularnego w~Qt wzorca Model-Widok~\cite{model-widok}. Może to być nieco uciążliwe dla początkujących programistów Qt, gdyż jest to obszerny system, a sprawne korzystanie z~niego wymaga doświadczenia.
Za prezentację danych odpowiada uniwersalna klasa dziedzicząca po QWidget.\newline

\textbf{Zalety:}
\begin{itemize}
\item{szeroki wybór wykresów,}
\item{zbiór wbudowanych interakcji oraz możliwość tworzenia własnych,}
\item{duże możliwości parametryzowania wyglądu wykresu,}
\item{wykresy 2,5D,}
\item{wysoka wydajność, umożliwiająca tworzenie wykresów czasu rzeczywistego.}\newline
\end{itemize}

\textbf{Wady:}
\begin{itemize}
\item{wysoka cena (jedno stanowisko -- 700 \euro),}
\item{wymuszenie korzystania z~modeli albo niskopoziomowych kontenerów do przechowywanie danych,}
\item{biblioteka napisana w~stylu Qt4, trudnym do wykorzystania w~QtQuick~2,}
\item{brak wsparcia dla QML.}
\end{itemize}

\subsection{QtitanChart}
Jest to biblioteka stworzona przez firmę Developer Machines~\cite{dev-machines}, udostępniająca dosyć duży duży zbiór wykresów. Twórcy biblioteki nie udostępniają pełnych źródeł swojej biblioteki, a~jedynie jej pliki nagłówkowe i~biblioteki dynamiczne, co znacznie utrudnia zrozumienie jej wewnętrznego działania. Ta wiedzia nie jest jednak potrzebna, gdyż aby tworzyć za jej pomocą wykresy, można się posługiwać komponentami wysokiego poziomu.\newline

Podobnie jak w Commercial Charts, logika wykresów jest zawarta w~seriach odpowiedniego typu.\newline

\textbf{Zalety:}
\begin{itemize}
\item{szeroki wybór wykresów,}
\item{możliwość wyeksportowania wykresu do pliku graficznego,}
\item{możliwość wyświetlania wykresów różnych typów w~jednym układzie współrzędnych,}
\item{motywy.}\newline
\end{itemize}

\textbf{Wady:}
\begin{itemize}
\item{wysoka cena (jedno stanowisko i~jedna platforma -- 519 \$),}
\item{brak wsparcia dla QML.}
\end{itemize}


\section{Rozwiązania darmowe} 
\subsection{Qwt}
Qt Widgets for Technical Applications to otwarta biblioteka umożliwiająca tworzenie wykresów oraz innych widgetów technicznych. Biblioteka ta była projektowana z~myślą o zastosowaniach technicznych, w~szczególności w~systemach czasu rzeczywistego, przez co głównym celem było zapewne osiągnięcie wysokiej wydajności przy dynamicznie zmieniających się danych. Cel ten osiągnięto między innymi poprzez wykorzystywanie stosunkowo niskopoziomowych mechanizmów, jak Arthur czy intensywne wykorzystanie szablonów języka C++. Niestety przez to biblioteka nie jest szczególnie przyjazna użytkownikowi. Programista często musi sięgać do niskopoziomowych narzędzia, a stworzenie mniej standardowego rozwiązania może sprawić wiele trudności.\newline

Qwt zawiera szerokie API służące do przeprowadzania różnorakich operacji na wykresach, m.in. skalowania i zaznaczania. Problematyczną kwestią jest wygląd wykresów, z~jednej strony są one adekwatne dla zastosowań technicznych, z~drugiej zaś sprawia on, że nie sposób użyć tych wykresów w~aplikacji innego typu, ze względu na ich brzydotę. Najlepszym przykładem niech będzie wykres załączony na rys. \ref{rys:wykres:sinus}, wyglądający niczym oscylogram.
\begin{figure}
\centering
\caption{Przykładowy wykres Qwt}\label{rys:wykres:sinus}
\includegraphics[scale=0.4]{img/sinus.png}
\end{figure}

\textbf{Zalety:}
\begin{itemize}
\item{rozbudowane API,}
\item{wsparcie społeczności,}
\item{wysoka wydajność,}
\item{skala logarytmiczna.}\newline
\end{itemize}

\textbf{Wady:}
\begin{itemize}
\item{nieprzyjazna użytkownikowi,}
\item{mało atrakcyjny wygląd wykresów,}
\item{brak wsparcia dla QML.}
\end{itemize}

\subsection{GobChartWidget}
Jest to jednoosobowy projekt rozwijany przez Williama Hallatta, dostępny na licencji open-source. Biblioteka ta ma bardzo ograniczoną funkcjonalność, pozwala na tworzenie wykresów tylko trzech typów: słupkowych, liniowych oraz kołowych, wszystkie o~bardzo prostym wyglądzie.\newline

Jest to kolejna biblioteka uzależniona od frameworku Model-Widok. Jest to jednak przypadek ekstremalny, nie polegający jedynie na trzymaniu danych w~modelu. Klasy odpowiedzialne za prezentację wykresów dziedziczą tu po abstrakcyjnym widoku z~ww. frameworku. Zapewne twórca uzyskał dzięki temu pewne ciekawe własności, jednak zmusiło go to do implementacji osobnych widoków dla każdego typu wykresu.\newline

Do rysowania wykresów wykorzystano tu GraphicsView, przy czym kontenerem na elementy nie jest scena, a~model.\newline

Jak widać GobChartWidget jest nieco dziwną hybrydą, której z~pewnością nie można nazwać skalowalną. Wcale nie dziwi fakt, że nie cieszy się ona szczególną popularnością wśród programistów Qt. Według statystyk SourceForge~\cite{forge}, w~ciągu pół roku pobrano tę bibliotekę zaledwie kilkanaście razy.

\textbf{Zalety:}
\begin{itemize}
\item{darmowa.}\newline
\end{itemize}

\textbf{Wady:}
\begin{itemize}
\item{ubogi zbiór wykresów,}
\item{mało przejrzysta struktura,}
\item{brak wspierającej społeczności programistów,}
\item{brak wsparcia dla QML.}
\end{itemize}



\section{Dostępne typy wykresów}
\begin{tabular}{|c|c|c|c|c|c|}
\hline
&  Qt Charts & KD Chart & Qtitan & Qwt & GobChart\\
\hline
Bąbelkowy & T & N & T & N & N\\
\hline
Gantta & N & T & N & N & N\\
\hline
Kołowy & T & T & T & N & T\\
\hline
Liniowy & T & T & T & T & T\\
\hline
Pierścieniowy & T & T & T & N & N\\
\hline
Słupkowy & T & T & T & N & T\\
\hline
Świecowy & T & T & N & N & N\\
\hline
Warstwowy & T & T & T & N & N\\
\hline
XY (punktowy) & T & N & T & T & N\\
\hline

\end{tabular}


\section{Elementy wspólne}
Analizując powyższe biblioteki dochodzimy do wniosku, że można wykroić z nich część wspólną, stanowiącą podstawową funkcjonalność niezbędną dla biblioteki tego typu. Te elementy to:
\begin{itemize}
\item{podstawowe wykresy, takie jak liniowy, słupkowy i kołowy,}
\item{osie, siatka i legenda,}
\item{możliwość realizacji przez programistę interakcji z~użytkownikiem,}
\item{zaznaczanie i przybliżanie fragmentów wykresu,}
\item{serie danych -- każda z bibliotek wykorzystuje ujednolicony interfejs do danych. Czesem seria jest jedynie opakowaniem na kontener z~próbkami, innym razem zawiera większość logiki związanej z~konstrukcją wykresu,}
\item{efekty 2,5-3D, komercyjne biblioteki umożliwiają tworzenie pseudo-przestrzennych wykresów.}
\end{itemize}

\section{Elementy unikalne}
Prawie wszystkie biblioteki zawierają wartościowe elementy niepowtarzalne lub rzadko spotykane:
\begin{itemize}
\item{animacja towarzysząca tworzeniu wykresów oraz ich przebudowywaniu,}
\item{motywy pozwalające na tworzenie wykresów w~różnych, jednolitych stylach,}
\item{możliwość realizacji pełnej interakcji ze wszystkimi elementami wykresu,}
\item{możliwość konfiguracji elementu prezentującego kolor w~legendzie (np. koło dla wykresu bąbelkowego, odcinek dla liniowego),}
\item{generowanie plików graficznych zawierających wykresy,}
\item{możliwość wyświetlania kilku wykresów różnego typu w~jednym układzie współrzędnych,}
\item{nieliniowe skale osi,}
\item{wyeksponowanie klas C++ w QML.}
\end{itemize}

\section{Podsumowanie}
Jak widać istnieją już na rynku wartościowe biblioteki pozwalające na tworzenie wykresów biurowych, są to jednak rozwiązania komercyjne. Projekty open-source są albo niskiej jakości albo mają nieco inne cele. Istnieje więc potrzeba stworzenia wysokiej jakości darmowego produktu, który zyskałby taką popularność jak Qwt. Potrzeba ta jest jeszcze większa dla QtQuick, gdzie nie ma żadnych darmowych bibliotek pozwalających na tworzenie wykresów.\newline

Warto również zauważyć, że twórcy żadnej z~opisanych bibliotek nie zdecydowali się na pełne odizolowanie silnika biblioteki od widoków. Przewiduję, że dzięki takiemu podejściu osiągnę możliwość wyświetlania wykresów w dowolnym miejscu -- wewnątrz widgetu, w~aplikacji napisanej w~QML czy w~dokumencie tekstowym. Byłoby to pierwsze takie rozwiązanie na rynku.





\begin{thebibliography}{}
\bibitem[1]{digia}
Strona domowa firmy Digia \url{http://digia.com}
\bibitem[2]{graphicsview}
GraphicsView \url{http://qt-project.org/doc/qt-5.0/qtwidgets/graphicsview.html}
\bibitem[3]{scenegraph}
SceneGraph \url{http://qt-project.org/doc/qt-5.0/qtquick/qtquick-visualcanvas-scenegraph.html}
\bibitem[4]{kdab}
Strona domowa firmy KDAB \url{http://kdab.com}
\bibitem[5]{arthur}
Arthur \url{http://qt-project.org/doc/qt-4.8/qt4-arthur.html}
\bibitem[6]{scribe}
Scribe \url{http://qt-project.org/doc/qt-4.8/qt4-scribe.html}
\bibitem[7]{model-widok}
Model-Widok \url{http://qt-project.org/doc/qt-5.0/qtwidgets/model-view-programming.html}
\bibitem[8]{dev-machines}
Developer Machines \url{http://www.devmachines.com}
\bibitem[9]{forge}
SourceForge \url{http://sourceforge.net}

\end{thebibliography}

