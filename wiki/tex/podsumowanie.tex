\chapter{Podsumowanie}
Praca inżynierska to pierwszy tak duży projekt, który przyszło mi realizować samodzielnie. Mimo, iż jej tytuł przewiduje projekt oraz implementację biblioteki, podczas realizacji pracowni dyplomowej czasu starczyło mi jedynie na stworzenie części projektowej oraz prototypu rozwiązania.

Ograniczona implementacja rozwiązania wynika z~nazbyt optymistycznej estymacji kosztów, głównie czasu potrzebnego na jej napisanie. Niedoszacowanie to wynika z~niedocenienia złożoności problemu. Prace były wydłużone przez chęć stworzenia rozwiązania jak najbardziej generycznego. Ogólność rozwiązania objawia się zarówno w~mnogości widoków jak i~źródeł danych kompatybilnych z~wykresami. Dość dużego narzutu pracy wymagało również uwzględnienie dostępności biblioteki w~Qt~Quick.

Mimo, iż układ pracy sugeruje kaskadowy model tworzenia oprogramowania, przebieg prac wyglądał nieco inaczej. Moje niewielkie doświadczenie w~zakresie projektowania oprogramowania oraz znajomość Qt na średnim poziomie lepiej korelowały z~modelem zwinnym i~właśnie tak starałem się tworzyć kolejne podrozdziały. Jednak układ pracy zgodny z~modelem kaskadowym jest bardziej logiczny i~spójny dla czytelnika.

Z~samego projektu biblioteki jestem dość zadowolony i~uważam, że wykonałem kluczową część pracy związanej ze stworzeniem biblioteki. Podejrzewam, że dowolny programista znający C++ oraz Qt byłby w~stanie zaimplementować rozwiązania zaprezentowane w~rozdziale \textit{Projekt}. Najbardziej zadowolony jestem z~wykorzystania umiejętności zdobytych podczas studiów oraz pogłębienia znajomości Qt.

Uważam, że zaproponowana przeze mnie architektura może być łatwo rozszerzana przez innych programistów. Nowe typy wykresów mogą korzystać z~gotowych rozwiązań i~dostosowywać je do swoich potrzeb. Główne kierunki rozwoju to dodawanie kolejnych typów wykresów oraz rozszerzenie zbioru interaktywnych operacji.

Podczas projektowania oraz implementacji biblioteki podjąłem wiele decyzji inżynierskich rzutujących na ostateczny kształ rozwiązania. Pozostaje mi jedynie mieć nadzieję, że większość z~nich podjąłem świadomie. Zapewne niektóre wybory nie były trafione, jednak uważam, że doświadczenie zdobyte podczas prac nad tym projektem zaowocuje w~przyszłości.


