\chapter{Implementacja}

\section{Narzędzia}
Poniżej opisuję narzędzia, które posłużą do stworzenia mojej biblioteki.

\subsection{Qt 5}
Premiera piątej wersji Qt miała miejsce 12 grudnia 2012. Aktualnie najnowsza dostępna wersja Qt to 5.1. Nowa odsłona dostarcza programistom szereg usprawnień oraz modułów, m.in. do obsługi formatu JSON. Jednak głównym punktem Qt~5 jest nowa implementacja Qt~Quick. 

Do implementacji Qt~Quick~2 wykorzystano OpenGL i~SceneGraph, co znacznie poprawiło wydajność tego systemu. Qt~5 rozpoczęło też nowy kierunek rozwoju aplikacji wykorzystujących Qt. Qt~Quick jest promowany jako zalecany sposób tworzenia interfejsów użytkownika. Docelowo aplikacje Qt mają być podzielone na GUI napisane w~QML oraz logikę zaprogramowaną w~C++.

\subsection{Qt Creator}
Qt~Creator to zintegrowane środowisko programistyczne przeznaczone głównie dla języków C++, QML oraz JavaScript. Jego edytor tekstowy zawiera takie udogodnienia jak kolorowanie składni czy narzędzia do refaktoryzacji kodu. Qt~Creator zawiera także wtyczkę do tworzenia graficznych interfejsów użytkownika. Korzystanie z~Designera jest proste i~intuicyjne, a~proste GUI można w~dużej mierze ,,wyklikać''.

Do budowania i~debugowania Qt~Creator wykorzystuje domyślne oprogramowanie danej platformy, np. kompilator gcc i~debuger gdb na systemie Linux. Creator posiada graficzny interfejs do debuggera, który w~znaczący sposób upraszcza procesz debugowania.

Qt~Creator posiada również wtyczki integrujące go z~najpopularniejszymi systemami kontroli wersji. Lista wspieranych systemów:
\begin{itemize}
\item{Bazaar}
\item{CVS}
\item{Git}
\item{Mercurial}
\item{Perforce}
\item{Subversion}
\end{itemize}

\subsection{Subversion}
Subversion, czyli w~skrócie SVN, to scentralizowany system kontroli wersji będący następcą systemu CVS. Repozytorium SVN założyłem w~serwisie Google Code~\footnote{http://code.google.com/intl/pl/}.

\subsection{Kubuntu 12.04 LTS}
Kubuntu to pochodna Ubuntu, korzystająca z~KDE~\footnote{KDE \url{http://pl.wikipedia.org/wiki/K\_Desktop\_Environment}} -- graficznego środowiska, zbudowanego w~oparciu o~bibliteki Qt. ,,Kubuntu oznacza \textit{w stronę ludzkości} w języku bemba''~\footnote{Kubuntu \url{http://pl.wikipedia.org/wiki/Kubuntu}} -- cytat ten jednoznacznie wskazuje co jest celem istnienia tej dystrybucji Linuxa.

Kubuntu jest udostępniane z~bogatym zbiorem aplikacji biurowych, multimedialnych oraz wielu innych. Najpopularniejsze aplikacje, które są dostarczane wraz z~systemem Kubuntu to LibreOffice i~GIMP. 

\section{Tworzenie biblioteki}

