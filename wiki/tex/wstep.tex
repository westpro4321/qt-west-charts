\part{Wstęp}
\section{Qt}
Qt jest zbiorem bibliotek języka C++, które najczęściej są rozpowszechniane jako biblioteki dynamiczne, jednak źródła Qt są udostępnione na stronie właściciela -- firmy Digia \cite{digia}. Lista bibliotek jest dość okazała, a~znaleźć na niej można narzędzia do tworzenia interfejsów użytkownika, parsowania plików XML czy dostępu do baz danych. Podstawowym celem twórców Qt było stworzenie przenośnych, wygodnych dla programisty oraz wydajnie działających bibliotek. Przenośność pomiędzy najpopularniejszymi platformami, osiągnięto na poziomie kodu źródłowego, a~nie skompilowanego programu. Naczelną zasadą Qt jest: \textit{pisz raz, kompiluj wielokrotnie}. 

\subsection{Motywacje do podjęcia tematu}
Jako, że nie chciałem tworzyć kolejnej aplikacji, która ostatecznie wyląduje w przysłowiowej szufladzie, zdecydowałem się na projekt biblioteki wykresów. Mimo iż w~ostatnim czasie na rynku pojawiło się kilka takich bibliotek, nadal, a może tym bardziej, warto stworzyć użyteczne narzędzie do tworzenia wykresów biurowych, które będzie konkurowało nie tylko z~darmowymi rozwiązaniami, ale nawet z~komercyjnymi produktami. Z przedstawionego w następnym rozdziale przeglądu dziedziny wynika, że istnieje sens tworzenia kolejnej biblioteki do operowania na wykresach. Musi ona jednak być wygodnym i uniwersalnym narzędziem oraz powinna udostępniać pogramiście więcej możliwości niż już dostępne, darmowe rozwiązania. Dodatkowo uwzględniając kierunek rozwoju Qt, warto zatroszczyć się o~interfejs dla Qt~Quick, aby stworzyć pierwszą taką bibliotekę dla tej technologii.


\section{API w stylu Qt}
Tworząc jakąkolwiek bibliotekę trzeba zadbać, aby jej API było czytelne i wygodne dla użytkownika. Dodatkowo, tworząc bibliotekę dla Qt, warto zaprojektować ją tak, aby dobrze wpisywała się w zbiór rozwiązań dostępnych w Qt. W jej strukturze nie może zabraknąć miejsca dla takich mechanizmów jak sygnały i~sloty czy właściwości. Krótko rzecz ujmując, moja biblioteka powinna uwzględniać podstawowe wytyczne podane pod adresem \url{http://doc.qt.digia.com/qq/qq13-apis.html}.

\section{Funkcjonalność}
Podstawową kwestią, którą trzeba poruszyć jest zakres funkcjonalności biblioteki. Jak wiele typów wykresów oraz jakie operacje na nich ma ona udostępniać?

\section{Nowoczesność a zgodność wsteczna}
Kolejnym problemem, przed jakim staję jest określenie jak nowoczesna może być to biblioteka. Czy biblioteka może działać z Qt5 i~nie udostępniać wsparcia dla Qt4? Jak może wpłynąć to na jej popularność? Czy korzystać z dobrodziejstw C++11, do którego kompilatorów nie wszyscy mają dostęp? Czy interfejs dla Qt~Quick ma być zgodny z~jego pierwszą czy aktualną wersją?

\section{Przenośność}
Jak wiadomo, podstawową zasadą Qt jest \textit{pisz raz, kompiluj wielokrotnie}. Jak bardzo przenośna ma być biblioteka? Czy w ramach pracowni dyplomowej wystarczy zapewnić zgodność z~rodziną systemów Windows NT oraz systemami Linux/UNIX? Ponadto, w jakiej formie najlepiej udostępniać bibliotekę, jako kompilowalne źródła czy może bibliotekę dynamiczną?

\section{Atrakcyjność, interakcyjność i optymalizacja}
Duży wpływ na popularność biblioteki wśród programistów będzie miał wygląd wykresów tworzonych za jej pomocą. Jak ważnym elementem jest estetyka wyglądu wykresów i czy powinna ona zdominować prace nad biblioteką? Czy potencjalnym klientom wystarczą płaskie wykresy, a może zechcą 2,5D (w stylu \textit{Graphics View}) lub pełnego 3D? Jak zapewnić tak wysoki poziom ogólności rozwiązania, że nie będzie ono ograniczać w przyszłości programistów, np. przy implementacji mechanizmu \textit{drag and drop}?
Jak wiele prac ma zostać poświęconych optymalizacji procesu tworzenia wykresów i~na jakim etapie ma się ona odbyć?


\section{Podsumowanie}
Te i~więcej pytań wymagają odpowiedzi przed przejściem do fazy projektowania. Część z nich pojawi się w~tekście zawierającym opis wymagań, natomiast inne zostaną rozwiązane dopiero na etapie analizy wymagań. Jednak samo ich postawienie zbliża mnie do utworzenia użytecznego narzędza do operowania na wykresach.

