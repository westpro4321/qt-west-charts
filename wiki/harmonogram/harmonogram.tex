\documentclass[11pt,twoside,a4paper,final]{article}
\usepackage[utf8]{inputenc}
\usepackage[T1]{fontenc}
\usepackage[MeX]{polski}
\usepackage{graphicx}
\usepackage{url}
\usepackage{hyperref}
\bibliographystyle{splncs}

\begin{document}

\date{16 czerwca 2013}
\title{Harmonogram prac}

\author{Łukasz Szewczyk}

\maketitle

\begin{tabular}{|c|c|p{8cm}|}
\hline
Termin & Element & Komentarz\\
\hline
30.06 & Przegląd dziedziny & Co prawda już go Panu wysyłałem, 
ale przejrzę to jeszcze raz i ewentualnie poprawię. \\
\hline
30.06 & Opis wymagań &  \\
\hline
14.07 & QML & Głębsze zapoznanie się z technologią Qt~Quick. \\
\hline
14.07 & Analiza wymagań & Analiza bardziej skomplikowanych wymagań(opis + ew. diagram use case). \\
\hline
28.07 & Zarys projektu & Pomysły realizacji zadania. Pierwsze diagramy. \\
\hline
28.07 & Plan testów & Poznanie QTestLib i sporządzenie planu testowania biblioteki. \\
\hline
11.08 & Końcowa faza projektu & Powinienem posiadać już wszystkie diagramy klas i~sekwencji. Na ich podstawie powinien powstać rozdział ,,Projekt''\\
\hline
25.08 & Implementacja & Prawdopodobnie szczątkowa implementacja wybranego wykresu + rozdział książki jej poświecony. \\
\hline
01.09 & Testy & Zawartość tego rozdziału jest mocno uzależniona od efektów implementacji.
Na pewno będzie zawierać opis QTestLib i~plan testów biblioteki.  \\
\hline
08.09 & Podsumowanie & Wnioski, itp.\\
\hline
\end{tabular}

\end{document}
