\documentclass[11pt,oneside,a4paper,final]{llncs}
\usepackage[utf8]{inputenc}
\usepackage[T1]{fontenc}
\usepackage[MeX]{polski}
\usepackage{graphicx}
\usepackage{url}
\usepackage{hyperref}
\bibliographystyle{splncs}

\begin{document}

\date{16 czerwca 2013}
\title{Spis treści}

\author{Łukasz Szewczyk}

\maketitle

\section{Wstęp}
\subsection{Qt}
\subsection{Opis problemu}


\section{Przegląd dziedziny}
\subsection{Opis rozwiązań darmowych}
\subsection{Opis rozwiązań komercyjnych}
\subsection{Elementy wspólne}
\subsection{Elementy unikalne}
\subsection{Podsumowanie\newline}

\section{Wymagania}
\subsection{Wymagania funkcjonalne}
\subsection{Wymagania pozafunkcjonalne}

\section{Analiza wymagań}
\subsection{Kolejne wymagania\newline}

\section{Projekt}
\subsection{Założenia}
\subsection{Struktura biblioteki}
\subsection{Diagramy klas}
\subsection{Diagramy sekwencji\newline}

\section{Implementacja}
\subsection{Opis wykorzystanych narzędzi}
\subsection{Szczegóły implementacyjne}
\subsection{Odstąpienia od założeń projektowych\newline}

\section{Testowanie i uruchamianie}
\subsection{Metody i narzędzia testowania}
\subsection{Przebieg testów}
\subsection{Wyniki i wnioski}
\subsection{Sprawdzone platformy uruchomieniowe\newline}

\section{Podsumowanie}
\subsection{Ocena jakości i~konkurencyjności stworzonej biblioteki}
\subsection{Wnioski dotyczące procesu projektowania i implementacji}
\subsection{Uwagi\newline}

\section{Bibliografia}

\section{Dodatki}
\subsection{Opis testowych aplikacji}
\subsection{Opis zawartości płyty dołączonej do pracy}

\end{document}
